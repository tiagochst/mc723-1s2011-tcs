\documentclass[10pt,a4paper]{article}
\usepackage[T1]{fontenc}
\usepackage[brazil]{babel}
\usepackage[utf8]{inputenc}


\usepackage{ae,aecompl}
\usepackage{pslatex}
\usepackage{epsfig}
\usepackage{geometry}
\usepackage{url}
\usepackage{textcomp}
\usepackage{ae}
\usepackage{subfig}
\usepackage{indentfirst}
\usepackage{textcomp}
\usepackage{color}
\usepackage{setspace}
\usepackage{verbatim}

\usepackage[compact]{titlesec}
\titlespacing{\section}{0pt}{*0}{*0}
\titlespacing{\subsection}{0pt}{*0}{*0}
\titlespacing{\subsubsection}{0pt}{*0}{*0}

%Definindo as margens para 2cm e o espaçamento entre linhas para 1.5
% Relatório parcial deve ter espaçamento simples
%\linespread{1.5}

\geometry{ 
	a4paper,	% Formato do papel
	tmargin=20mm,	% Margem superior
	bmargin=20mm,	% Margem inferior
	lmargin=20mm,	% Margem esquerda
	rmargin=20mm,	% Margem direita
	footskip=20mm	% Espaço entre o rodapé e o fim do texto
}
\renewcommand{\thetable}{\Roman{table}}
\newcommand{\x} {$\bullet$}


\begin{document}


\setlength{\parskip}{0pt}
\setlength{\parsep}{0pt}
\setlength{\headsep}{0pt}
\setlength{\topskip}{0pt}
\setlength{\topmargin}{0pt}
\setlength{\topsep}{0pt}
\setlength{\partopsep}{0pt}


% Guia para o relatório
%Quais são os principais parâmetros a serem definidos em uma cache?
%Quais são os valores típicos para esses parâmetros?
%Quais devem ser os limites mínimos e máximos desses valores?
%Olhe no manual do dinero e descubra quais desses parâmetros ele
%permite configurar -- Done
%Como podemos dizer que uma determinada configuração de cache é melhor que outra?
%O que é um trace de execução? -- Done +-
%Por que utilizar um trace de execução para achar a melhor configuração de cache para um programa?
%Por que escolher a melhor configuração de cache para um dado programa? A configuração de cache não é específica do processador?
%Olhe aqui qual o programa e número de arquivos você deve usar
\begin{minipage}{5cm}
\makebox[5cm][l]{\rule{4cm}{1mm}{\hspace{4mm}\bf Tiago Chedraoui Silva \hspace{2mm}RA:082941
 \hspace{2mm} Turma: A}\hspace{4mm} \rule{4cm}{1mm} }
\vspace{2mm}
\end{minipage}



\section{Conclusão}



% ******************************************************
% 		REFERENCIAS BIBLIOGRÁFICAS
% ******************************************************
%\section{Referências}
\bibliographystyle{plain}
\begin{small}
  \bibliography{referencias}
\end{small}

\end{document}
