\documentclass[10pt,a4paper]{article}
\usepackage[T1]{fontenc}
\usepackage[brazil]{babel}
\usepackage[utf8]{inputenc}


\usepackage{ae,aecompl}
\usepackage{pslatex}
\usepackage{epsfig}
\usepackage{geometry}
\usepackage{url}
\usepackage{textcomp}
\usepackage{ae}
\usepackage{subfig}
\usepackage{indentfirst}
\usepackage{textcomp}
\usepackage{color}
\usepackage{setspace}
\usepackage{verbatim}
\renewcommand{\thetable}{\Roman{table}}
\usepackage{hyperref}

%\usepackage[compact]{titlesec}
%\titlespacing{\section}{0pt}{*0}{*0}
%\titlespacing{\subsection}{0pt}{*0}{*0}
%\titlespacing{\subsubsection}{0pt}{*0}{*0}

%Definindo as margens para 2cm e o espaçamento entre linhas para 1.5
% Relatório parcial deve ter espaçamento simples
%\linespread{1.5}

\geometry{ 
	a4paper,	% Formato do papel
	tmargin=20mm,	% Margem superior
	bmargin=20mm,	% Margem inferior
	lmargin=20mm,	% Margem esquerda
	rmargin=20mm,	% Margem direita
	footskip=20mm	% Espaço entre o rodapé e o fim do texto
}


\include{abaco} 

\hypersetup{
    bookmarks=true,         % show bookmarks bar?
    unicode=false,          % non-Latin characters in Acrobat’s bookmarks
    pdftoolbar=true,        % show Acrobat’s toolbar?
    pdfmenubar=true,        % show Acrobat’s menu?
    pdffitwindow=false,     % window fit to page when opened
    pdfstartview={FitH},    % fits the width of the page to the window
    pdftitle={My title},    % title
    pdfauthor={Author},     % author
    pdfsubject={Subject},   % subject of the document
    pdfcreator={Creator},   % creator of the document
    pdfproducer={Producer}, % producer of the document
    pdfkeywords={keyword1} {key2} {key3}, % list of keywords
    pdfnewwindow=true,      % links in new window
    colorlinks=true,       % false: boxed links; true: colored links
    linkcolor=blue,          % color of internal links
    citecolor=blue,        % color of links to bibliography
    filecolor=white,      % color of file links
    urlcolor=blue           % color of external links
}
\begin{document}

% CAPA
  \thispagestyle{empty}
  
  \begin{minipage}[h]{0.10\linewidth}
    \ABACO{1}{9}{6}{9}{0.5} 
  \end{minipage}
  \begin{minipage}[h!]{0.7\linewidth}
    \vspace*{\fill}
    \centering
    {\large \textbf{UNIVERSIDADE~ESTADUAL~DE~CAMPINAS}}\\ 
    {\large INSTITUTO~DE~COMPUTAÇÃO}                   
    \vspace*{\fill} 
  \end{minipage}
    \\\vspace{0.5cm}
  
  \begin{center} 
    \rule{11.0cm}{0.4pt}\vspace*{-\baselineskip}\vspace{-2.0pt}
    \rule{11.0cm}{1.6pt} \vspace*{2.0pt}\\
      {\Large \textsc{Modelagem de processadores em ArchC}}\vspace{-3.2pt}
    \rule{11.0cm}{0.4pt}\vspace*{-\baselineskip}\vspace{3.2pt} \rule{11.0cm}{1.6pt}\\
    {\textsl{Proposta de projeto}}
    \\\vspace{1cm}
    \begin{tabular}{ll}
      \textbf{Nome}: Douglas Alves Germano        &   \textbf{RA}: 060210\\
      \textbf{Nome}: Tiago Chedraoui Silva        &       \textbf{RA}: 082941
      
    \end{tabular}
  \end{center}
  \vspace{0.5cm}
  
  \section*{Introdução}
  
  O archC é uma linguagem de descrição de arquiteturas livre
  desenvolvido no Laboratório de Sistemas de Computação (LSC) na
  Unicamp. Utilizando essa linguagem desenvolveremos um simulador de um sistema dedicado. 
  
  \section*{Descrição}
  Modelaremos o processador da familia \href{http://cache.freescale.com/files/microcontrollers/doc/ref_manual/RS08RM.pdf}{RS08}\cite{FS} da empresa freescale na
  linguagem ArchC.
  
  \section*{Próximas fases}
  
  \begin{description}
  \item[Fase 2] Composta pelos item 1, 2 e 3 da lista do cronograma:
    Iniciaresmos a implementação do processador em ArchC.
  \item[Fase 3] Composta pelo item 4 lista do cronograma: Terminaremos
    a implementação.
 
  \end{description}
  
  \section*{Cronograma}
    
  \begin{enumerate}
  \item Instalar localmente ArchC
  \item Estudar como desenvolver um processador na linguagem ArchC
  \item Implementar código com funções básicas 
  \item Implementação código com a maioria das funções presentes no processador
    
  \end{enumerate}

  \begin{table}[h!]
    \caption{Cronograma das atividades}
    \begin{center} {
        \begin{tabular}{ |c|c|c|c|c|c|c|}
          \hline
          & \multicolumn{3}{|c|}{Fase 2} & \multicolumn{3}{|c|}{Fase 3} 	\\ \hline
          Semana & 01 & 02  & 03 & 04 & 05 & 06 \\ \hline
          1 & $\bullet$ &  & &  & & \\ \hline
          2 & $\bullet$ & $\bullet$ & & &  & \\ \hline
          3 & & $\bullet$ & $\bullet$ & &  & \\ \hline
          4 &  &  &  & $\bullet$ & $\bullet$ & $\bullet$\\ \hline
          

          \hline
        \end{tabular}
      }\end{center}
    \label{tab:cronograma}
  \end{table}

 \addcontentsline{toc}{section}{Referências}
 \bibliographystyle{unsrtbr}
 \begin{small}\bibliography{../../bibliografia/referencias}\end{small}
 \label{referencias}
  
\end{document}