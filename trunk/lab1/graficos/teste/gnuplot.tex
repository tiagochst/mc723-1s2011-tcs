\documentclass[10pt,a4paper]{article}
\usepackage[T1]{fontenc}
\usepackage[brazil]{babel}
\usepackage[utf8]{inputenc}


\usepackage{ae,aecompl}
\usepackage{pslatex}
\usepackage{epsfig}
\usepackage{geometry}
\usepackage{url}
\usepackage{textcomp}
\usepackage{ae}
\usepackage{subfig}
\usepackage{indentfirst}
\usepackage{textcomp}
\usepackage{color}
\usepackage{setspace}
\usepackage{verbatim}

%Definindo as margens para 2cm e o espaçamento entre linhas para 1.5
% Relatório parcial deve ter espaçamento simples
%\linespread{1.5}

\geometry{ 
	a4paper,	% Formato do papel
	tmargin=20mm,	% Margem superior
	bmargin=20mm,	% Margem inferior
	lmargin=20mm,	% Margem esquerda
	rmargin=20mm,	% Margem direita
	footskip=20mm	% Espaço entre o rodapé e o fim do texto
}
\renewcommand{\thetable}{\Roman{table}}
\newcommand{\x} {$\bullet$}

% Graficos
\usepackage[miktex]{gnuplottex}
\usepackage{tikz}
\usepackage{gnuplot-lua-tikz}
\usepackage{mathpazo}

\begin{document}
%% GNUPLOT: LaTeX picture with Postscript
\begingroup
  \makeatletter
  \providecommand\color[2][]{%
    \GenericError{(gnuplot) \space\space\space\@spaces}{%
      Package color not loaded in conjunction with
      terminal option `colourtext'%
    }{See the gnuplot documentation for explanation.%
    }{Either use 'blacktext' in gnuplot or load the package
      color.sty in LaTeX.}%
    \renewcommand\color[2][]{}%
  }%
  \providecommand\includegraphics[2][]{%
    \GenericError{(gnuplot) \space\space\space\@spaces}{%
      Package graphicx or graphics not loaded%
    }{See the gnuplot documentation for explanation.%
    }{The gnuplot epslatex terminal needs graphicx.sty or graphics.sty.}%
    \renewcommand\includegraphics[2][]{}%
  }%
  \providecommand\rotatebox[2]{#2}%
  \@ifundefined{ifGPcolor}{%
    \newif\ifGPcolor
    \GPcolorfalse
  }{}%
  \@ifundefined{ifGPblacktext}{%
    \newif\ifGPblacktext
    \GPblacktexttrue
  }{}%
  % define a \g@addto@macro without @ in the name:
  \let\gplgaddtomacro\g@addto@macro
  % define empty templates for all commands taking text:
  \gdef\gplbacktext{}%
  \gdef\gplfronttext{}%
  \makeatother
  \ifGPblacktext
    % no textcolor at all
    \def\colorrgb#1{}%
    \def\colorgray#1{}%
  \else
    % gray or color?
    \ifGPcolor
      \def\colorrgb#1{\color[rgb]{#1}}%
      \def\colorgray#1{\color[gray]{#1}}%
      \expandafter\def\csname LTw\endcsname{\color{white}}%
      \expandafter\def\csname LTb\endcsname{\color{black}}%
      \expandafter\def\csname LTa\endcsname{\color{black}}%
      \expandafter\def\csname LT0\endcsname{\color[rgb]{1,0,0}}%
      \expandafter\def\csname LT1\endcsname{\color[rgb]{0,1,0}}%
      \expandafter\def\csname LT2\endcsname{\color[rgb]{0,0,1}}%
      \expandafter\def\csname LT3\endcsname{\color[rgb]{1,0,1}}%
      \expandafter\def\csname LT4\endcsname{\color[rgb]{0,1,1}}%
      \expandafter\def\csname LT5\endcsname{\color[rgb]{1,1,0}}%
      \expandafter\def\csname LT6\endcsname{\color[rgb]{0,0,0}}%
      \expandafter\def\csname LT7\endcsname{\color[rgb]{1,0.3,0}}%
      \expandafter\def\csname LT8\endcsname{\color[rgb]{0.5,0.5,0.5}}%
    \else
      % gray
      \def\colorrgb#1{\color{black}}%
      \def\colorgray#1{\color[gray]{#1}}%
      \expandafter\def\csname LTw\endcsname{\color{white}}%
      \expandafter\def\csname LTb\endcsname{\color{black}}%
      \expandafter\def\csname LTa\endcsname{\color{black}}%
      \expandafter\def\csname LT0\endcsname{\color{black}}%
      \expandafter\def\csname LT1\endcsname{\color{black}}%
      \expandafter\def\csname LT2\endcsname{\color{black}}%
      \expandafter\def\csname LT3\endcsname{\color{black}}%
      \expandafter\def\csname LT4\endcsname{\color{black}}%
      \expandafter\def\csname LT5\endcsname{\color{black}}%
      \expandafter\def\csname LT6\endcsname{\color{black}}%
      \expandafter\def\csname LT7\endcsname{\color{black}}%
      \expandafter\def\csname LT8\endcsname{\color{black}}%
    \fi
  \fi
  \setlength{\unitlength}{0.0500bp}%
  \begin{picture}(7200.00,5040.00)%
    \gplgaddtomacro\gplbacktext{%
      \csname LTb\endcsname%
      \put(946,704){\makebox(0,0)[r]{\strut{}$3$}}%
      \put(946,1317){\makebox(0,0)[r]{\strut{}$4$}}%
      \put(946,1929){\makebox(0,0)[r]{\strut{}$5$}}%
      \put(946,2542){\makebox(0,0)[r]{\strut{}$6$}}%
      \put(946,3154){\makebox(0,0)[r]{\strut{}$7$}}%
      \put(946,3767){\makebox(0,0)[r]{\strut{}$8$}}%
      \put(946,4379){\makebox(0,0)[r]{\strut{}$9$}}%
      \put(1078,484){\makebox(0,0){\strut{}$1$}}%
      \put(2043,484){\makebox(0,0){\strut{}$2$}}%
      \put(3008,484){\makebox(0,0){\strut{}$3$}}%
      \put(3973,484){\makebox(0,0){\strut{}$4$}}%
      \put(4939,484){\makebox(0,0){\strut{}$5$}}%
      \put(5904,484){\makebox(0,0){\strut{}$6$}}%
      \put(6869,484){\makebox(0,0){\strut{}$7$}}%
      \put(308,2541){\rotatebox{-270}{\makebox(0,0){\strut{}Voltage $V_p$ [V]}}}%
      \put(3973,154){\makebox(0,0){\strut{}Resistance $R_0$ [$Omega$]}}%
      \put(3973,4709){\makebox(0,0){\strut{}Graph 3: Dependence of $V_p$ on $R_0$}}%
    }%
    \gplgaddtomacro\gplfronttext{%
      \csname LTb\endcsname%
      \put(5882,4206){\makebox(0,0)[r]{\strut{}"graph1.csv" using 1:3}}%
    }%
    \gplbacktext
    \put(0,0){\includegraphics{graph1}}%
    \gplfronttext
  \end{picture}%
\endgroup

 \begin{gnuplot}[scale=0.95,terminal=lua tikz]
     set parametric
     set angle degree
     set urange [0:360]
     set vrange [-90:90]
     fx(u,v)=cos(u)*cos(v)
     fy(u,v)=sin(u)*cos(v)
     fz(v)=sin(v)
     splot fx(u,v),fy(u,v),fz(v)
 \end{gnuplot}
\end{document}
