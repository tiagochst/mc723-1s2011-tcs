\documentclass[12pt,a4paper]{article}
\usepackage[T1]{fontenc}
\usepackage[brazil]{babel}
\usepackage[utf8]{inputenc}


\usepackage{ae,aecompl}
\usepackage{pslatex}
\usepackage{epsfig}
\usepackage{geometry}
\usepackage{url}
\usepackage{textcomp}
\usepackage{ae}
\usepackage{subfig}
\usepackage{indentfirst}
\usepackage{textcomp}
\usepackage{color}
\usepackage{setspace}
\usepackage{verbatim}



%Definindo as margens para 2cm e o espaçamento entre linhas para 1.5
% Relatório parcial deve ter espaçamento simples
%\linespread{1.5}

\geometry{ 
	a4paper,	% Formato do papel
	tmargin=25mm,	% Margem superior
	bmargin=25mm,	% Margem inferior
	lmargin=20mm,	% Margem esquerda
	rmargin=20mm,	% Margem direita
	footskip=20mm	% Espaço entre o rodapé e o fim do texto
}
\include{abaco} 
\renewcommand{\thetable}{\Roman{table}}
\newcommand{\x} {$\bullet$}


\begin{document}

% CAPA
\begin{titlepage}
\thispagestyle{empty}
  \begin{center} {\large \textbf{UNIVERSIDADE~ESTADUAL~DE~CAMPINAS}} \end{center}
  \begin{center} {\large INSTITUTO~DE~COMPUTAÇÃO}                    \end{center}
  \vspace{0.1cm}
  \begin{center}
  \begin{minipage}[tl]{31mm}
    \ABACO{1}{9}{6}{9}{1}
  \end{minipage}
  \end{center}
  \vspace{0.3cm}
  \begin{center} 
    {\large \textsc{Estudo da cache através de simulações
               }} 
    \\\vspace{0.5cm}
    {\textsl{Relatório do primeiro laboratório de MC723}}
    \\\vspace{1cm}
    \begin{tabular}{rl}
	  \textbf{Aluno}:       & Tiago~Chedraoui~Silva \\
	\end{tabular}
  \end{center}
  \vspace{0.5cm}

  \begin{abstract}
O princípio de funcionamento da memória cache é duplicar parte dos dados contidos na memória
principal (a memória lenta, neste caso) em um módulo menor (o cache) composto por dispositivos
de memória mais rápidos.
Quando o processador solicita um item de dado (gerando uma referência para seu endereço, que
pode ser físico ou virtual), o gerenciador de memória requisita este item do cache. Duas situações
podem ocorrer:
cache hit: item está presente no cache, é retornado para o processador praticamente sem período de
latência;
cache miss: item não está presente no cache, processador deve aguardar item ser buscado da memó-
ria principal.
Nesse laboratório, estudaremos a melhor organização de uma memória
cache para a execução de um determinado programa.

  \end{abstract}

  % Sumário
  \tableofcontents
\end{titlepage} 


% Guia para o relatório
%Quais são os principais parâmetros a serem definidos em uma cache?
%Quais são os valores típicos para esses parâmetros?
%Quais devem ser os limites mínimos e máximos desses valores?
%Olhe no manual do dinero e descubra quais desses parâmetros ele
%permite configurar -- Done
%Como podemos dizer que uma determinada configuração de cache é melhor que outra?
%O que é um trace de execução? -- Done +-
%Por que utilizar um trace de execução para achar a melhor configuração de cache para um programa?
%Por que escolher a melhor configuração de cache para um dado programa? A configuração de cache não é específica do processador?
%Olhe aqui qual o programa e número de arquivos você deve usar

%-----------------------------------------------------------------------------%
\section{Resumo das atividades}
%-----------------------------------------------------------------------------%

\subsection{Dinero}
%Olhe no manual do dinero e descubra quais desses parâmetros ele permite configurar

O Software Dinero é um silmulador de cache para traces de memória
(registro de execução de um programa).

Dentre as opções de configuração da memória cache, o dinero nos
fornece as seguintes possibilidades:
tamanho da memória chace, tamanho do bloco da memória
cache, tamanho do sub-bloco, associatividade, política de
substituição, política de escrita se occorer um hit (write-back, write
through), política de escrita se occorer um
miss (write-allocate,no-write-allocate,fectch,non-fetch).

% -lN-Tsize P       Size
% -lN-Tbsize P      Block size
% -lN-Tsbsize P     Sub-block size (default same as block size)
% -lN-Tassoc U      Associativity (default 1)
% -lN-Trepl C       Replacement policy
%                   (l=LRU, f=FIFO, r=random) (default l)
% -lN-Tfetch C      Fetch policy
%                   (d=demand, a=always, m=miss, t=tagged,
%                    l=load forward, s=subblock) (default d)
% -lN-Twalloc C     Write allocate policy
%                   (a=always, n=never, f=nofetch) (default a)
% -lN-Twback C      Write back policy
%                   (a=always, n=never, f=nofetch) (default a)

\end{document}