\documentclass[10pt,a4paper]{article}
\usepackage[T1]{fontenc}
\usepackage[brazil]{babel}
\usepackage[utf8]{inputenc}


\usepackage{ae,aecompl}
\usepackage{pslatex}
\usepackage{epsfig}
\usepackage{geometry}
\usepackage{url}
\usepackage{textcomp}
\usepackage{ae}
\usepackage{subfig}
\usepackage{indentfirst}
\usepackage{textcomp}
\usepackage{color}
\usepackage{setspace}
\usepackage{verbatim}
\usepackage{hyperref}

% Gráficos
% \usepackage{pgfplots}
% \pgfplotsset{compat=1.3}
% \usepgfplotslibrary{groupplots}

\usepackage{hyperref}
\hypersetup{
  colorlinks,%
  citecolor=black,%
  filecolor=black,%
  linkcolor=blue,%
  urlcolor=blue
}

\usepackage[compact]{titlesec}
\titlespacing{\section}{0pt}{*0}{*0}
\titlespacing{\subsection}{0pt}{*0}{*0}
\titlespacing{\subsubsection}{0pt}{*0}{*0}

% Definindo as margens para 2cm e o espaçamento entre linhas para 1.5
% Relatório parcial deve ter espaçamento simples
% \linespread{1.5}

\geometry{ 
  a4paper,	% Formato do papel
  tmargin=40mm,	% Margem superior
  bmargin=30mm,	% Margem inferior
  lmargin=20mm,	% Margem esquerda
  rmargin=20mm,	% Margem direita
  footskip=20mm	% Espaço entre o rodapé e o fim do texto
}
\renewcommand{\thetable}{\Roman{table}}
\newcommand{\x} {$\bullet$}


\begin{document}


\setlength{\parskip}{0pt}
\setlength{\parsep}{0pt}
\setlength{\headsep}{0pt}
\setlength{\topskip}{0pt}
\setlength{\topmargin}{0pt}
\setlength{\topsep}{0pt}
\setlength{\partopsep}{0pt}


% Guia para o relatório
% Quais são os principais parâmetros a serem definidos em uma cache?
% Quais são os valores típicos para esses parâmetros?
% Quais devem ser os limites mínimos e máximos desses valores?
% Olhe no manual do dinero e descubra quais desses parâmetros ele
% permite configurar -- Done
% Como podemos dizer que uma determinada configuração de cache é melhor que outra?
% O que é um trace de execução? -- Done +-
% Por que utilizar um trace de execução para achar a melhor configuração de cache para um programa?
% Por que escolher a melhor configuração de cache para um dado programa? A configuração de cache não é específica do processador?
% Olhe aqui qual o programa e número de arquivos você deve usar
\begin{minipage}{5cm}
  \makebox[5cm][l]{\rule{4cm}{1mm}{\hspace{4mm}\bf Tiago Chedraoui Silva \hspace{2mm}RA:082941
      \hspace{2mm} Turma: A}\hspace{4mm} \rule{4cm}{1mm} }
  \vspace{2mm}
\end{minipage}

\vspace{-3mm}

\section{Parte 1}
Para que serve a opção -n do make?

R:A opção -n serve para esconder os warnings.

Quantas regras existem no Makefile do lsf/encoder?

R:Existem 4 regras no Makefile do lsf/encoder.
\subsection{Parte 1 - compilação}

\begin{table}[h!]
  \caption{Função que mais gastam tempo no programa enconde}
\begin{center}
  \begin{tabular}{ccccccc}

Otimização& tempo real& tempo user& tempo sys\\
\hline 
Nenhuma  &  real	0m0.572s & user	0m0.290s& sys	0m0.244s\\

-O0&
real	0m3.618s&
user	0m2.445s&
sys	0m0.651s\\

-O1&
real	0m6.076s&
user	0m4.793s&
sys	0m0.694s\\

 -O2&
real	0m7.775s&
user	0m6.472s&
sys	0m0.740s\\

-O3&
real	0m9.882s&
user	0m8.337s&
sys	0m0.727s\\

 \end{tabular}
\end{center}
\end{table}

-mtune= cpu\underline{}type 
Otimização compilação de acordo com arquitetura

\begin{table}[h!]
  \caption{Usando flag mtune em compilação}
\begin{center}
  \begin{tabular}{ccccccc}
Otimização& tempo real& tempo user& tempo sys\\
\hline
make &
real	0m3.560s&
user	0m2.471s&
sys	0m0.625s\\

make -O2&
real	0m7.614s&
user	0m6.471s&
sys	0m0.690s\\

make -O3&
real	0m9.761s&
user	0m8.185s&
sys	0m0.784s


 \end{tabular}
\end{center}
\end{table}


% -mtune=name
% This option is very similar to the -mcpu= option, except that instead of specifying the actual target processor type, and hence restricting
% which instructions can be used, it specifies that GCC should tune the performance of the code as if the target were of the type specified in
% this option, but still choosing the instructions that it will generate based on the cpu specified by a -mcpu= option.  For some ARM
% implementations better performance can be obtained by using this option.

% -mtune=architecture-type
% Tune to architecture-type everything applicable about the generated code, except for the ABI and the set of available instructions.  The
% choices for architecture-type are the same as for -march=architecture-type.



% -mtune=cpu-type
% Tune to cpu-type everything applicable about the generated code, except for the ABI and the set of available instructions.  The choices for
% cpu-type are:

% generic
% Produce code optimized for the most common IA32/AMD64/EM64T processors.  If you know the CPU on which your code will run, then you should
% use the corresponding -mtune option instead of -mtune=generic.  But, if you do not know exactly what CPU users of your application will
% have, then you should use this option.

% As new processors are deployed in the marketplace, the behavior of this option will change.  Therefore, if you upgrade to a newer version
% of GCC, the code generated option will change to reflect the processors that were most common when that version of GCC was released.

% There is no -march=generic option because -march indicates the instruction set the compiler can use, and there is no generic instruction
% set applicable to all processors.  In contrast, -mtune indicates the processor (or, in this case, collection of processors) for which the
% code is optimized.

% native
% This selects the CPU to tune for at compilation time by determining the processor type of the compiling machine.  Using -mtune=native will
% produce code optimized for the local machine under the constraints of the selected instruction set.  Using -march=native will enable all
% instruction subsets supported by the local machine (hence the result might not run on different machines).

% i386
% Original Intel's i386 CPU.

% i586, pentium
% Intel Pentium CPU with no MMX support.

% pentium-mmx
% Intel PentiumMMX CPU based on Pentium core with MMX instruction set support.

% pentiumpro
% Intel PentiumPro CPU.

% i686
% Same as "generic", but when used as "march" option, PentiumPro instruction set will be used, so the code will run on all i686 family
% chips.

% core2
% Intel Core2 CPU with 64-bit extensions, MMX, SSE, SSE2, SSE3 and SSSE3 instruction set support.

% atom
% Intel Atom CPU with 64-bit extensions, MMX, SSE, SSE2, SSE3 and SSSE3 instruction set support.


% athlon, athlon-tbird
% AMD Athlon CPU with MMX, 3dNOW!, enhanced 3DNow! and SSE prefetch instructions support.

% athlon-4, athlon-xp, athlon-mp
% Improved AMD Athlon CPU with MMX, 3DNow!, enhanced 3DNow! and full SSE instruction set support.

% k8, opteron, athlon64, athlon-fx
% AMD K8 core based CPUs with x86-64 instruction set support.  (This supersets MMX, SSE, SSE2, 3DNow!, enhanced 3DNow! and 64-bit
% instruction set extensions.)


% While picking a specific cpu-type will schedule things appropriately for that particular chip, the compiler will not generate any code that
% does not run on the i386 without the -march=cpu-type option being used.


\section{Parte 2}
O valor do vetor off após a décima primeira execução da função
window_subband é:

\section{Parte 3}
Exacutando:
\begin{verbatim}
 ./encode ../../../heavydrum.aiff 
 gprof encode  > withOX.stats
\end{verbatim}
Sendo X o número da flag -OX passada para o gcc encontramos as duas
funções que mais gastam tempo:

\begin{table}[h!]
  \caption{Função que mais gastam tempo no programa enconde}
  \begin{tabular}{ccccccc}
    Otimização & Função 1 & Tempo(\%) & Tempo(s) & Função 2  & Tempo(\%) &
    Tempo(s)\\
    \hline\\
    O0 & filter\underline{ }subband &42.51 \% &1.59 s &
    II\underline{ }f\underline{ }f\underline{ }t& 13.10 \%& 0.49 s\\
    O1 & filter\underline{ }subband & 44.96 \%& 1.65 s &
    add\underline{ }db& 10.63\%& 0.39 s\\
    O2 &
    filter\underline{ }subband& 21.50 \%& 0.63 s &
    add\underline{ }db &15.63\% &0.45 s\\
    O3 &
    filter\underline{ }subband &24.62\%& 0.65 s&
    threshold & 16.17\%& 0.44 s


  \end{tabular}
\end{table}

\section{Parte 4}
Meu usuário no github (git) https://github.com/tiagochst/\\
Meu usuário no google code (svn) http://code.google.com/u/@VxBQQFNXDhRBWAR6/
\section{Parte 5}

% ******************************************************
% REFERENCIAS BIBLIOGRÁFICAS
% ******************************************************
% \section{Referências}
% \bibliographystyle{plain}
% \begin{small}
%   \bibliography{referencias}
% \end{small}

\end{document}
